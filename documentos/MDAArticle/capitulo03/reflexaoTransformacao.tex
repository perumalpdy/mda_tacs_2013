\section{Reflexão Transformação}

	Nesta seção segue a reflexão inicial.

\subsection{Avaliação do código gerado}

	O código foi gerado conforme o modelo. Nota-se a existência de algumas anotações da ferramenta o que prejudica um pouco a leitura do código. Para os tipos múltiplos, a ferramenta gera um tipo List parametrizado pelo tipo em questão. A ferramenta gera um método getter privado para uma propriedade privada (configuração, falta inibição?).

\subsection{Utilidade prática desta ferramenta}

	A utilidade prática é de aumentar a produtividade com a geração automática do código através da diagramação.

\subsection{Elementos faltantes}

\begin{tabular}{| p{5cm} | p{4cm} | p{5cm} |}
  \hline
  \textbf{Elemento(s)} & \textbf{Diagrama(s) UML Correspondente(s)} & \textbf{Grau de Dificuldade de Transformação Automática} \\
	\hline
  Sequencia mínima de interação entre as classes & Diagrama de Sequência & ? \\
	\hline
  Definição da plataforma específica & &  \\
	\hline
  Importação de bibliotecas estrangeiras & Diagrama de Componentes & ? \\
	\hline
  Definição de Pacotes & Diagrama de Classes & ? \\
	\hline
\end{tabular}