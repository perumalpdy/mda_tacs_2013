%--------------------------------------------------
% Estudo de caso MDA
%--------------------------------------------------

\documentclass[11pt]{article}

% Configurações

\usepackage[utf8]{inputenc}
\usepackage{geometry}
\usepackage{graphicx}

\geometry{a4paper}

\usepackage{booktabs}
\usepackage{array}
\usepackage{paralist}
\usepackage{verbatim}
\usepackage{subfig}
\usepackage{color}
\usepackage{listings}
\usepackage{fancyhdr}
\pagestyle{fancy}
\renewcommand{\headrulewidth}{0pt}
\lhead{}\chead{}\rhead{}
\lfoot{}\cfoot{\thepage}\rfoot{}

\usepackage{sectsty}
\allsectionsfont{\sffamily\mdseries\upshape}

\usepackage[nottoc,notlof,notlot]{tocbibind}
\usepackage[titles,subfigure]{tocloft}
\renewcommand{\cftsecfont}{\rmfamily\mdseries\upshape}
\renewcommand{\cftsecpagefont}{\rmfamily\mdseries\upshape}

\usepackage{graphicx}
\usepackage{pdflscape}
\usepackage{caption}

\captionsetup[lstlisting]{font={small,tt}}

\lstset{ %
language=Java,                % choose the language of the code
basicstyle=\footnotesize,       % the size of the fonts that are used for the code
numbers=left,                   % where to put the line-numbers
numberstyle=\footnotesize,      % the size of the fonts that are used for the line-numbers
stepnumber=1,                   % the step between two line-numbers. If it is 1 each line will be numbered
numbersep=5pt,                  % how far the line-numbers are from the code
backgroundcolor=\color{white},  % choose the background color. You must add \usepackage{color}
showspaces=false,               % show spaces adding particular underscores
showstringspaces=false,         % underline spaces within strings
showtabs=false,                 % show tabs within strings adding particular underscores
frame=single,           % adds a frame around the code
tabsize=4,          % sets default tabsize to 2 spaces
captionpos=b,           % sets the caption-position to bottom
breaklines=true,        % sets automatic line breaking
breakatwhitespace=false,    % sets if automatic breaks should only happen at whitespace
escapeinside={\%*}{*)}          % if you want to add a comment within your code
}

\lstset{frameround=fttt}

% Fim das configurações
% Início do documento

\title{Pizzaria Geek: Um estudo MDA}
\author{Paulo Ortolan \and André Nucci \and Paulo Fernandes}

\begin{document}
\maketitle

\section{Descrição do Projeto}

A pizzaria Geek, famosa por abrigar nerds, geeks e fãs de RPGs como Dungeons \& Dragons e GURPS, quer melhorar seu sistema, criado em uma feira medieval no verão de 87 (a.k.a. Clipper 87) para algo mais moderno como o sistema LCARS da Nova Geração da Enterprise.

No entanto, o mestre dos magos da pizzaria é cético quanto à essa modernização. Ele não acha que um sistema pode ser feito como armaduras do Iron Man. Sua crença se resume à frases de efeitos do Sr. Miyagi e conselhos inversos do Yoda (Anos 80 isso é Hmmmmm!). Ele quer ser convencido que suas pizzas mais saborosas, como a Obi Wan Calzone ou Ham Solo, vão ser vendidas melhor com o novo sistema.

Este relatório tem como objetivo mostrar que isso é possível e não está sob domínio exclusivo de onerosos estúdios de Hollywood ou fazem parte do imaginário da Ficção Científica.

\section{Diagrama de Classes}

Nesta seção, seguem os diagramas de classes gerados pela ferramenta Modelio.

\begin{figure}[h!]
 \centering
 \includegraphics[scale=0.5]{capitulo02/diagramaClassesCadastrarAdicionais.png}
 \caption{Diagrama de Classes Cadastrar Adicionais}
\end{figure}

\begin{figure}[h!]
 \centering
 \includegraphics[scale=0.5]{capitulo02/diagramaClassesCadastrarClientes.png}
 \caption{Diagrama de Classes Cadastrar Adicionais}
\end{figure}

\begin{figure}[h!]
 \centering
 \includegraphics[scale=0.5]{capitulo02/diagramaClassesCadastrarPedidos.png}
 \caption{Diagrama de Classes Cadastrar Pedidos}
\end{figure}

\begin{figure}[h!]
 \centering
 \includegraphics[scale=0.5]{capitulo02/diagramaClassesCadastrarPizzas.png}
 \caption{Diagrama de Classes Cadastrar Pizzas}
\end{figure}

\begin{figure}[h!]
 \centering
 \includegraphics[scale=0.5]{capitulo02/diagramaClassesCadastrarPromocoes.png}
 \caption{Diagrama de Classes Cadastrar Promoções}
\end{figure}

\begin{figure}[h!]
 \centering
 \includegraphics[scale=0.5]{capitulo02/diagramaClassesEmitirRelatorioEstoque.png}
 \caption{Diagrama de Classes Emitir Relatório de Estoque}
\end{figure}

\begin{figure}[h!]
 \centering
 \includegraphics[scale=0.5]{capitulo02/diagramaClassesEmitirRelatorioPedido.png}
 \caption{Diagrama de Classes Emitir Relatório de Pedido}
\end{figure}

\begin{figure}[h!]
 \centering
 \includegraphics[scale=0.5]{capitulo02/diagramaClassesReportBuilder.png}
 \caption{Diagrama de Classes Report Builder}
\end{figure}

\begin{landscape}
\begin{figure}[h!]
 \centering
 \includegraphics[scale=0.5]{capitulo02/diagramaClassesCadastrarUsuarios.png}
 \caption{Diagrama de Classes Cadastrar Usuários}
\end{figure}
\end{landscape}

\begin{landscape}
\begin{figure}[h!]
 \centering
 \includegraphics[scale=0.5]{capitulo02/diagramaClassesMER.png}
 \caption{Diagrama de Classes Modelo Entidade Relacionamento}
\end{figure}
\end{landscape}


\section{Reflexão Transformação}

	Nesta seção segue a reflexão inicial.

\subsection{Avaliação do código gerado}

	O código foi gerado conforme o modelo. Nota-se a existência de algumas anotações da ferramenta o que prejudica um pouco a leitura do código. Para os tipos múltiplos, a ferramenta gera um tipo List parametrizado pelo tipo em questão. A ferramenta gera um método getter privado para uma propriedade privada (configuração, falta inibição?).

\subsection{Utilidade prática desta ferramenta}

	A utilidade prática é de aumentar a produtividade com a geração automática do código através da diagramação.

\subsection{Elementos faltantes}

\begin{tabular}{| p{5cm} | p{4cm} | p{5cm} |}
  \hline
  \textbf{Elemento(s)} & \textbf{Diagrama(s) UML Correspondente(s)} & \textbf{Grau de Dificuldade de Transformação Automática} \\
	\hline
  Sequencia mínima de interação entre as classes & Diagrama de Sequência & ? \\
	\hline
  Definição da plataforma específica & &  \\
	\hline
  Importação de bibliotecas estrangeiras & Diagrama de Componentes & ? \\
	\hline
  Definição de Pacotes & Diagrama de Classes & ? \\
	\hline
\end{tabular}

\section{Código gerado}

Nesta seção segue a listagem do código gerado pela ferramenta Modelio.
\newline

\lstinputlisting[language=Java, caption=Entidade mapeada do banco de dados para salvar dados dos itens adicionais]{capitulo04/Adicional.java}
\lstinputlisting[language=Java, caption=Classe de regras de negócio para os itens adicionais]{capitulo04/AdicionalBusiness.java}
\lstinputlisting[language=Java, caption=Interface para comunicação com banco de dados para a entidade Adicional]{capitulo04/AdicionalRepository.java}
\lstinputlisting[language=Java, caption=Classe para interação com o usuário final para CRUD de itens adicionais]{capitulo04/AdicionalView.java}
\lstinputlisting[language=Java, caption=Entidade mapeada do banco de dados para salvar dados dos cliente]{capitulo04/Cliente.java}
\lstinputlisting[language=Java, caption=Classe de regras de negócio para o cliente]{capitulo04/ClienteBusiness.java}
\lstinputlisting[language=Java, caption=Interface para comunicação com banco de dados para a entidade Cliente]{capitulo04/ClienteRepository.java}
\lstinputlisting[language=Java, caption=Classe para interação com o usuário final]{capitulo04/ClienteView.java}
\lstinputlisting[language=Java, caption=Entidade mapeada do banco de dados para salvar dados dos cliente]{capitulo04/Combo.java}
\lstinputlisting[language=Java, caption=Interface para comunicação com banco de dados para a entidade Adicional]{capitulo04/ComboRepository.java}
\lstinputlisting[language=Java, caption=Interface genérica para itens consumíveis\, tais como Pizza\, Adicionais e Combos]{capitulo04/Consumivel.java}
\lstinputlisting[language=Java, caption=Classe de negócio para consultar serviços dos Correios]{capitulo04/ECTBusiness.java}
\lstinputlisting[language=Java, caption=Serviço que implementa cliente Web Service para consultar serviços do Correio]{capitulo04/ECTService.java}
\lstinputlisting[language=Java, caption=Entidade para guardar informações de endereços obtidos da classe ECTService]{capitulo04/EnderecoECT.java}
\lstinputlisting[language=Java, caption=Classe de regras de negócio para relatório de estoque]{capitulo04/EstoqueBusiness.java}
\lstinputlisting[language=Java, caption=Classe para interação com o usuário final para emissão de relatórios de Estoque]{capitulo04/EstoqueView.java}
\lstinputlisting[language=Java, caption=Classe que guarda um item de pedido e sua quantidade]{capitulo04/ItemDePedido.java}
\lstinputlisting[language=Java, caption=Classe responsável por criar uma mensagem de e-mail]{capitulo04/MailModel.java}
\lstinputlisting[language=Java, caption=Classe responsável por fazer a interface com ]{capitulo04/MailSender.java}
\lstinputlisting[language=Java, caption=Entidade mapeada do banco de dados para salvar dados dos pedidos]{capitulo04/Pedido.java}
\lstinputlisting[language=Java, caption=Classe de regras de negócio para os pedidos]{capitulo04/PedidosBusiness.java}
\lstinputlisting[language=Java, caption=Interface para comunicação com banco de dados para a entidade Pedido]{capitulo04/PedidosRepository.java}
\lstinputlisting[language=Java, caption=Classe para interação com o usuário final para CRUD de pedidos]{capitulo04/PedidosView.java}
\lstinputlisting[language=Java, caption=Entidade mapeada do banco de dados para salvar dados das pizzas]{capitulo04/Pizza.java}
\lstinputlisting[language=Java, caption=Classe de regras de negócio para as pizzas]{capitulo04/PizzaBusiness.java}
\lstinputlisting[language=Java, caption=Interface para comunicação com banco de dados para a entidade Pizza]{capitulo04/PizzaRepository.java}
\lstinputlisting[language=Java, caption=Classe para interação com o usuário final para CRUD de pizzas]{capitulo04/PizzaView.java}
\lstinputlisting[language=Java, caption=Entidade mapeada do banco de dados para salvar dados das promoções]{capitulo04/Promocao.java}
\lstinputlisting[language=Java, caption=Classe de regras de negócio para as promoções]{capitulo04/PromocaoBusiness.java}
\lstinputlisting[language=Java, caption=Interface para comunicação com banco de dados para a entidade Promocao]{capitulo04/PromocaoRepository.java}
\lstinputlisting[language=Java, caption=Classe para interação com o usuário final para CRUD de promoções]{capitulo04/PromocaoView.java}
\lstinputlisting[language=Java, caption=Classe para a criação de relatórios]{capitulo04/ReportBuilder.java}
\lstinputlisting[language=Java, caption=Enum para tipos de adicionais]{capitulo04/TipoAdicional.java}
\lstinputlisting[language=Java, caption=Enum para tipos de promoções]{capitulo04/TipoPromocao.java}
\lstinputlisting[language=Java, caption=Entidade mapeada do banco de dados para salvar dados de usuários]{capitulo04/Usuario.java}
\lstinputlisting[language=Java, caption=Classe de regras de negócio para os usuários]{capitulo04/UsuarioBusiness.java}
\lstinputlisting[language=Java, caption=Interface para comunicação com banco de dados para a entidade Usuario]{capitulo04/UsuarioRepository.java}
\lstinputlisting[language=Java, caption=Classe para interação com o usuário final para CRUD de usuários]{capitulo04/UsuarioView.java}

\section{Código Sintaxe PL/0}

Nesta seção segue a listagem do código sintático em Java para a linguagem PL/0.
\newline

\lstinputlisting[language=Java, caption=Classe Java para análise sintática da linguagem PL/0]{capitulo05/Sintatico.java}

\section{Código de uma classe Java}

Listagem da calsse PedidosView.java.
\newline
\begin{lstlisting}
package br.senac.sp.pizzariageek.view;

import br.senac.sp.pizzariageek.business.PedidosBusiness;
import br.senac.sp.pizzariageek.entities.Adicional;
import br.senac.sp.pizzariageek.entities.Cliente;
import br.senac.sp.pizzariageek.entities.Combo;
import br.senac.sp.pizzariageek.entities.Pedido;
import br.senac.sp.pizzariageek.entities.Pizza;
import br.senac.sp.pizzariageek.entities.Promocao;
import br.senac.sp.pizzariageek.utils.ReportBuilder;
import java.util.Date;
import java.util.List;

public class PedidosView {
    private PedidosBusiness pedidosBusiness;

    private PedidosBusiness getPedidosBusiness() {
        return this.pedidosBusiness;
    }

    private ReportBuilder reportBuilder;

    private ReportBuilder getReportBuilder() {
        return this.reportBuilder;
    }

    public Cliente procurarClientePorTelefone(final String telefone) {
        return pedidosBusiness.procurarClientePorTelefone(telefone);
    }

    public Cliente procurarClientePorEndereco(final String endereco) {
        return pedidosBusiness.procurarClientePorEndereco(endereco);
    }

    public boolean adicionarAdicional(final Adicional adicional) {
        pedidosBusiness.adicionarItem(adicional);
        return true;
    }

    public boolean adicionarPizza(final Pizza pizza) {
        pedidosBusiness.adicionarItem(pizza);
        return true;
    }

    public Pedido finalizarPedido() {
        return pedidosBusiness.finalizarPedido();
    }

    public boolean iniciarPedido() {
        pedidosBusiness.iniciarPedido();
        return true;
    }

    public void popularListaAdicionais() {
        pedidosBusiness.obterTodosAdicionais();
    }

    public void popularListaPizzas() {
        pedidosBusiness.obterTodasPizzas();
    }

    public List<Promocao> popularListaPromocoes() {
        return pedidosBusiness.obterTodasPromocoes();
    }

    public List<Combo> popularListaCombo() {
        return pedidosBusiness.obterTodosCombos();
    }

    public void adicionarCombo(final Combo combo) {
        pedidosBusiness.adicionarItem(combo);
    }

    public byte procurarPorPeriodo(final Date dataInicio, final Date dataFinal, final String tipoRelatorio) {
        return 0;
    }

}
\end{lstlisting}

Listagem da classe PedidosBusiness.java.
\newline
\begin{lstlisting}
package br.senac.sp.pizzariageek.business;

import br.senac.sp.pizzariageek.entities.Adicional;
import br.senac.sp.pizzariageek.entities.Cliente;
import br.senac.sp.pizzariageek.entities.Combo;
import br.senac.sp.pizzariageek.entities.Consumivel;
import br.senac.sp.pizzariageek.entities.ItemDePedido;
import br.senac.sp.pizzariageek.entities.Pedido;
import br.senac.sp.pizzariageek.entities.Pizza;
import br.senac.sp.pizzariageek.entities.Promocao;
import br.senac.sp.pizzariageek.repository.AdicionalRepository;
import br.senac.sp.pizzariageek.repository.ClienteRepository;
import br.senac.sp.pizzariageek.repository.ComboRepository;
import br.senac.sp.pizzariageek.repository.PedidosRepository;
import br.senac.sp.pizzariageek.repository.PizzaRepository;
import java.util.ArrayList;
import java.util.Date;
import java.util.List;

public class PedidosBusiness {
    private PedidosRepository pedidosRepository;
    private ClienteRepository clienteRepository;
    private PizzaRepository pizzaRepository;
    private AdicionalRepository adicionalRepository;
    private ComboRepository comboRepository;
    private PromocaoBusiness promocaoBusiness;
    private Cliente cliente;
    private Pedido pedido;
    private List<ItemDePedido> itens;
    private double valorTotal = new Double(0);
    
    private static final Double QUANTIDADE_UM = new Double(1);

    public Cliente procurarClientePorTelefone(final String telefone) {
        cliente = clienteRepository.findByTelefone(telefone);
        return cliente;
    }

    public Cliente procurarClientePorEndereco(final String endereco) {
        cliente = clienteRepository.findByEndereco(endereco);
        return cliente;
    }

    public void adicionarItem(final Consumivel item) {
        ItemDePedido itemDePedido = new ItemDePedido();
        
        itemDePedido.setConsumivel(item);
        itemDePedido.setQuantidade(QUANTIDADE_UM);
        
        itens.add(itemDePedido);
    }

    public double calcularTotal() {
        for(ItemDePedido item : itens) {
            valorTotal += item.getTotal();
        }
        
        return valorTotal;
    }

    public void iniciarPedido() {
        itens = new ArrayList<ItemDePedido>();
    }

    public Pedido finalizarPedido() {
        Pedido pedido = new Pedido();
        
        pedido.setCliente(cliente);
        pedido.setItems(itens);
        pedido.setValorTotal(calcularTotal());
        
        return pedido;
    }

    public List<Adicional> obterTodosAdicionais() {
        return adicionalRepository.listAll();
    }

    public List<Pizza> obterTodasPizzas() {
        return pizzaRepository.listAll();
    }

    public void aplicarPromocao(final long valor, final Promocao Promocao) {
    }

    public List<Combo> obterTodosCombos() {
        return comboRepository.listAll();
    }

    public List<Promocao> obterTodasPromocoes() {
        return promocaoBusiness.obterListaPromocoes();
    }

    public List<Pedido> listaPedidosPorPeriodo(final Date dataInicial, final Date dataFinal) {
        return null;
    }

}
\end{lstlisting}

\section{Diagramas Adicionais do Projeto}

Nesta seção, seguem os diagramas adicionais do projeto.

\subsection{Diagrama de Atividades}

\begin{figure}[h!]
 \centering
 \includegraphics[scale=0.6]{capitulo07/diagramaAtividades.png}
 \caption{Diagrama de Atividades Cadastrar Pizza}
\end{figure}

\newpage
\subsection{Diagramas de Sequência}

\begin{figure}[h!]
 \centering
 \includegraphics[scale=0.6]{capitulo07/diagramaSequenciasCadastrarPizzas.png}
 \caption{Diagrama de Sequência Cadastrar Pizza}
\end{figure}

\begin{landscape}
\begin{figure}[h!]
 \centering
 \includegraphics[scale=0.30]{capitulo07/digramaSequenciasSolicitarPedido.png}
 \caption{Diagrama de Sequência Solicitar Pedidos}
\end{figure}
\end{landscape}

\begin{landscape}
  \section{Mapeamento}
	\subsection{Elementos Válidos}
	\begin{table}[ht]
		\centering
		\begin{tabular}{|m{2cm}|m{2cm}|m{4cm}|m{4cm}|m{7cm}|}
			\hline
			
			\textbf{Número} &
			\textbf{Nome} & 
			\textbf{Descrição} &
			\textbf{Exemplo/Gráfico} &
			\textbf{Exemplo de Codificação} \\
			
			\hline
			
			1 &
			Diagrama de Classes/Classe & 
			Nome, Tipo (Classe, Classe Abstrata, Interface, Enum) &
			\includegraphics{capitulo08/ClassName.png} & 
			\begin{verbatim}
			public class $ClassName {
			}
			\end{verbatim}
			\\

			\hline
			2 &
			Diagrama de Classes / Atributos & 
			Acessor, tipo e nome &
			\includegraphics{capitulo08/Type.png} & 
			\begin{verbatim}
			$accessor $Type $type;
			\end{verbatim}
			\\

			\hline
			3 &
			Diagrama de Classes / Cardinalidade & 
			Um para um, um para muitos &
			\includegraphics{capitulo08/Collection.png} & 
			\begin{verbatim}
			private $Collection<Type> $type;
			\end{verbatim}
			\\

			\hline
			4 &
			Diagrama de Classes / Construtor & 
			Método público especial com o mesmo nome da classe que tem a responsabilidade de criar uma instância da classe &
			\includegraphics[scale=0.75]{capitulo08/Constructor.png} & 
			\begin{verbatim}
			public $ClassName([$params]);
			\end{verbatim}
			\\
			
			\hline
			5 &
			Diagrama de Classes / Método & 
			Método é uma operação que a classe realiza &
			\includegraphics[scale=0.75]{capitulo08/Method.png} & 
			\begin{verbatim}
			$accessor $returnType 
			  $method([$params]);
			\end{verbatim}
			\\
			
			\hline
		\end{tabular}
	\end{table}
	
	\newpage
	
	\subsection{Estruturas Válidas}
	\begin{table}[ht]
		\centering
		\begin{tabular}{|m{2cm}|m{2cm}|m{4cm}|m{5.5cm}|m{5.5cm}|}
			\hline
			
			\textbf{Número} &
			\textbf{Nome} & 
			\textbf{Descrição} &
			\textbf{Sequência / Interligação Gráfico} &
			\textbf{Exemplo de Codificação} \\
			
			\hline
			
			1 &
			Getter & 
			Método acessor para obter valores de um atributo encapsulado em uma classe &
			\includegraphics{capitulo08/Getter.png} & 
			\begin{verbatim}
			public $Type get$Type() {
				return this.$type;
			}
			\end{verbatim}
			\\

			\hline
			2 &
			Setter & 
			Método acessor para atribuir valores à um atributo encapsulado em uma classe &
			\includegraphics[scale=0.75]{capitulo08/Setter.png} & 
			\begin{verbatim}
			public void set$Type
					($Type $shadowType) {
				this.$type = $shadowType;
			}
			\end{verbatim}
			\\

			\hline
			3 &
			Condicional & 
			A partir de uma condição de guarda o método decide qual direção seguir. &
			\includegraphics[scale=0.5]{capitulo08/If.png} & 
			\begin{verbatim}
			if($parm == $guard) {
				doNormalFlow();
			} else {
				doAlternateFlow();
			}
			\end{verbatim}
			\\
			
			\hline
		\end{tabular}
	\end{table}
\end{landscape}

\subsection{Regras}
\begin{table}[ht]
	\centering
	\begin{tabular}{|m{2cm}|m{3cm}|m{6cm}|m{4cm}|}
		\hline
		
		\textbf{Número} &
		\textbf{Nome} & 
		\textbf{Descrição} &
		\textbf{Ilustração} \\
			
		\hline
		1 &
		Nome da Classe &
		Nome da classe seguindo o estilo de codificação sa SUN &
		\begin{verbatim}$ClassName\end{verbatim}
		\\

		\hline
		2 &
		Tipo &
		Tipo do atributo&
		\begin{verbatim}$Type\end{verbatim}
		\\

		\hline
		3 &
		Membro &
		Nome do atributo &
		\begin{verbatim}$type\end{verbatim}
		\\

		\hline
		4 &
		Getter &
		Método para obter um valor de um atributo &
		\begin{verbatim}get$Type\end{verbatim}
		\\
			
		\hline
		5 &
		Setter &
		Método para atribuir valores aos atributos &
		\begin{verbatim}set$Type\end{verbatim}
		\\

		\hline
		6 &
		Coleção &
		Coleção de objetos, pode ser mapeada para uma lista ordenada ou um conjunto &
		\begin{verbatim}$Collection\end{verbatim}
		\\

		\hline
		7 &
		Tipo Sombra &
		Parâmetro com o mesmo nome e tipo de um atributo da classe usado para transportar valores para um atributo interno da classe. &
		\begin{verbatim}$shadowType\end{verbatim}
		\\

		\hline
		8 &
		Acessor &
		Tipo de acesso à atributos. Pode assumir: private, public, protected ou "" (default) &
		\begin{verbatim}$accessor\end{verbatim}
		\\
			
		\hline
		9 &
		Método &
		Nome do método &
		\begin{verbatim}$method\end{verbatim}
		\\

		\hline
		10 &
		Lista de Parâmetros &
		Listagem de parâmetros para um método &
		\begin{verbatim}$params\end{verbatim}
		\\

		\hline
		11 &
		Tipo de Retorno &
		Tipo de Retorno que um método pode ter &
		\begin{verbatim}$returnType\end{verbatim}
		\\
		
		\hline
	\end{tabular}
\end{table}


\section{Autômato Sintático Classes Java}

ASDF

\end{document}

% Fim do documento